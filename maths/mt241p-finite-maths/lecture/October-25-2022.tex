\documentclass[12pt]{article} % use the article class

%%% The following looks horrible, but essentially sets up biblatex for producing
%%% bibliographies that look nice (you likely will not even need this)
\usepackage[style=authoryear-comp, maxcitenames=2, isbn=false, url=false,
  giveninits=true, doi=false, eprint=false, dashed=false, date=year,
  related=false, mergedate=true]{biblatex}
\renewbibmacro{in:}{%
  \ifboolexpr{%
    test {\ifentrytype{article}}%
    or
    test {\ifentrytype{inproceedings}}%
  }{}{\printtext{\bibstring{in}\intitlepunct}}%
}
\renewbibmacro*{cite:labelyear+extrayear}{%
  \ifentrytype{online}
  {}
  {\iffieldundef{labelyear}
    {}
    {\printtext[bibhyperref]{%
        \printfield{labelyear}
        \printfield{extrayear}}}}}

\AtEveryBibitem{%
  \clearfield{pagetotal}%
}

\DeclareFieldFormat[article,inbook,incollection,techreport,inproceedings,patent,thesis,unpublished]{title}{#1\isdot}
\DeclareFieldFormat{pages}{#1}

\usepackage[plain]{algorithm} % algorithms package
\usepackage{algpseudocode} % pseudo-code package
\usepackage{amsbsy} % for producing bold maths symbols
\usepackage{amsfonts} % an extended set of fonts for maths
\usepackage{amssymb} % various maths symbols
\usepackage{amsthm} % for producing theorem-like environments
\usepackage{blindtext}
\usepackage{datetime2} % managing dates and times
\usepackage{delimseasy} % makes easy the manual sizing of brackets, square brackets, and curly brackets
\usepackage{enumitem} % customing list environments
\usepackage{extramarks} % extra marks
\usepackage{fancyhdr} % headers and footers
\usepackage{float} % makes dealing with floats (e.g. tables and figures) easier
\usepackage{framed} % for producing framed boxes
\usepackage{graphicx} % for including graphics in the document
\usepackage{hyperref} % automatically produce hyperlinks for cross-references
\usepackage{import} % Import and subimport
\usepackage{listings} % Code blocks
\usepackage{mathtools} % package for maths (fixes some deficiences of amsmath so is preferred)
\usepackage{mdframed} % boxes
\usepackage{microtype} % better font sizing (extremely helpful with long equations!)
\usepackage{multicol}
\usepackage{newtx} % a fonts package
\usepackage{parskip} % Remove indents and separate paragraphs
\usepackage{pdfpages} % for including pdf documents inside the compiled pdf
\usepackage{pgf} % produce pdf graphics using LaTeX
\usepackage{pgfplots} % create normal/logarithmic plots in two and three dimensions
\pgfplotsset{compat=1.18} % sorts out the compatability warning
\usepackage{physics} % useful for vector calculus and linear algebra symbols
\usepackage[most]{tcolorbox} % for producing coloured boxes
\tcbuselibrary{theorems} % theorems with tcolorbox
\usepackage{tikz-3dplot} % for producing 3d plots
\usepackage{tikz} % for drawing graphics in LaTeX
\usepackage{tkz-base} % drawing with a Cartesian coordinate system
\usepackage{tkz-euclide} % drawing in Euclidean geometry
\usepackage{xcolor} % a package for colours

\usetikzlibrary{automata, positioning}

\DeclareMathAlphabet{\mathcal}{OMS}{cmsy}{m}{n}
 % imports the packages.tex file from the preamble directory
%%% Define the homeworkProblem environment
\newcommand{\enterProblemHeader}[1]{
    \nobreak\extramarks{}{Problem \arabic{#1} continued on next page\ldots}\nobreak{}
    \nobreak\extramarks{Problem \arabic{#1} (continued)}{Problem \arabic{#1} continued on next page\ldots}\nobreak{}
}

\newcommand{\exitProblemHeader}[1]{
    \nobreak\extramarks{Problem \arabic{#1} (continued)}{Problem \arabic{#1} continued on next page\ldots}\nobreak{}
    \stepcounter{#1}
    \nobreak\extramarks{Problem \arabic{#1}}{}\nobreak{}
}

\setcounter{secnumdepth}{0}
\newcounter{partCounter}
\newcounter{homeworkProblemCounter}
\setcounter{homeworkProblemCounter}{1}
\nobreak\extramarks{Problem \arabic{homeworkProblemCounter}}{}\nobreak{}

\newenvironment{homeworkProblem}[1][-1]{
    \ifnum#1>0
        \setcounter{homeworkProblemCounter}{#1}
    \fi
    \section{Problem \arabic{homeworkProblemCounter}}
    \setcounter{partCounter}{1}
    \enterProblemHeader{homeworkProblemCounter}
}{
    \exitProblemHeader{homeworkProblemCounter}
}
 % imports the environments.tex file from the preamble directory
\renewcommand{\part}[1]{\textbf{\large Part \Alph{partCounter}}\stepcounter{partCounter}\\} % part macro
\newcommand{\solution}{\textbf{\large Solution}} % solution macro

\newmdenv[
    backgroundcolor=gray!20,
    skipabove=\topsep,
    skipbelow=\topsep,
]{grayBoxed}

% General writing
\newcommand{\bracket}[1]{\left(#1\right)} % for automatic resizing of brackets
\newcommand{\sbracket}[1]{\left[#1\right]} % for automatic resizing of square brackets
\newcommand{\mset}[1]{\left\{#1\right\}} % for automatic resizing of curly brackets
\newcommand{\defeq}{\coloneqq} % the "defined as" command
\newcommand{\RNum}[1]{\uppercase\expandafter{\romannumeral #1\relax}} % uppercase roman numerals

\newcommand{\rednote}[1]{{\color{red} #1}} % for red text
\newcommand{\bluenote}[1]{{\color{blue} #1}} % for blue text
\newcommand{\greennote}[1]{{\color{green} #1}} % for green text

% Blackboard Maths Symbols
\newcommand{\N}{\mathbb{N}} % natural numbers
\newcommand{\Q}{\mathbb{Q}} % rational numbers
\newcommand{\Z}{\mathbb{Z}} % integers
\newcommand{\R}{\mathbb{R}} % real numbers
\newcommand{\C}{\mathbb{C}} % complex numbers
\newcommand{\E}{\mathbb{E}} % expectation operators

% Aesthetic
\newcommand{\thmBox}[2]{
    \begin{tcolorbox}[colback=blue!5!white,colframe=blue!50!black,
            colbacktitle=blue!90!black,title=Theorem #1]
        #2
    \end{tcolorbox}
}

\newcommand{\lmBox}[2]{
    \begin{tcolorbox}[colback=blue!5!white,colframe=blue!50!black,
            colbacktitle=blue!50!black,title=Lemma #1]
        #2
    \end{tcolorbox}
}

\newcommand{\defBox}[2]{
    \begin{tcolorbox}[colback=red!5!white,colframe=red!50!black,
            colbacktitle=red!60!black,title=Definition #1]
        #2
    \end{tcolorbox}
}

% Homework Specific
\newcommand{\hBox}[1]{
    \begin{tcolorbox}[colback=blue!5!white,colframe=blue!50!black]
        #1
    \end{tcolorbox}
} % imports the macros.tex file from the preamble directory
%%% DOCUMENT SETTINGS
\topmargin=-0.45in
\evensidemargin=0in
\oddsidemargin=0in
\textwidth=6.5in
\textheight=9.0in
\headsep=0.25in
\linespread{1.1}
\pagestyle{fancy}
\lhead{\hmwkAuthorName}
\chead{}
\rhead{\hmwkStudentnum}
\lfoot{\lastxmark}
\cfoot{\thepage}
\renewcommand\headrulewidth{0.4pt}
\renewcommand\footrulewidth{0.4pt}
\setlength\parindent{0pt}


%%% HOMEWORK DETAILS
\newcommand{\hmwkClassInstructor}{LECTURER NAME AND TITLE}
\newcommand{\hmwkAuthorName}{\textbf{Angel Cervera Roldan}}
 % imports the settings.tex file from the preamble directory
 

\title{
    \vspace{2in}
        \textmd{\textbf{MT231 - Finite Maths}}\\
    \vspace{1in}
    \textmd{\textbf{Lecture Notes}}\\
    \vspace{1in}
}

\author{
    \hmwkAuthorName\\
    \hmwkStudentnum
}

\date{25th October 2022}

\begin{document}

\maketitle

\pagebreak

Sieve of Eratosthenes:

Given $n \in \N$, we can define the primes in $S \defeq \mset{2, ..., n}$. Then for a $m \in S$, either m is a prime, or
$m = pr$, where $p, r \in S$. We can let the minimal in $S$ be $p$ by the well ordering principle. Then $p \leq r$. Now $p^2 \leq p r = m$, and so

$$
    p \leq \sqrt{m} \leq \sqrt{n}
$$

Overall, m is a prime or m is divisible by an integer p with $2 \leq p \leq \sqrt{n}$.

\subsection*{Lemma 5.3}
\begin{grayBoxed}
    For integers a,b, let p be a prime divisor of ab. Then $p | a$ or $p | b$
\end{grayBoxed}

If $p|a$, then there is nothing to prove. So assume that $p \not | a$. Then $gcd(p, a) = 1$. We deduce from Euclid's lemma that $p | b$

\subsection*{Corollary 5.4}
Let $p, a_1, a_2, ..., a_k$ be integers where p is prime. If $p | (a_1 \cdot a_2\cdot ...\cdot a_k)$ then $p|a_i$ for some $i \in \mset{1, 2, ..., k}$

\subsection*{Corollary 5.5}
Let $q_1, q_2, ..., q_k$ be a prime integer if $p | (q_1 \cdot q_2 \cdot ... \cdot q_k)$, then $p = q_i$ for some i.

\subsection*{Theorem 5.6}
\begin{grayBoxed}
    Fundamental theorem of arithmetic

    Given $n \in \Z$, non-zero, there exists $\varepsilon \in {\pm 1}$ and primes $p_1, ..., p_k$ such that:

    $$
        n = \varepsilon \cdot p_1 \cdot ... \cdot p_k
    $$
\end{grayBoxed}


Without loss of generality, we assume that $n \geq 1$.

The statement holds for $n = 1$. You would choose k to be 0, therefore no prime numbers will be selected, and $\varepsilon = 1$.

Now we assume that it holds for some $n \in \Z$.

In the case that $n$ is a prime,then we choose $\varepsilon = 1$, and we are done.

Otherwise, there is some positive divisor, say m, of n such that $m \not \in \mset{1, n}$. Then $n = m r$, for some $r \in \Z$. Note that $1 < m, r < n$.
By assumption, m and r are products of prime integers, and thus so is n. The uniqueness of this expression can be shown using Corollary 5.5.s So the statement holds
for n, and so by induction it holds for all $n \geq 1$

\subsection*{Corollary 5.8}
\begin{grayBoxed}
    There is an infinite number of prime integers
\end{grayBoxed}

Suppose $p_1, ..., p_n$ are all prime integers.
$$
    q = 1 + \prod_{i = 1}^{n} p_i
$$

q must be an integer which is not divisible by any of the $p$, hence q is a `new' prime number by theorem 5.6.

\subsection*{Remark 5.9}

\subsubsection*{1}
Let $a, b \in \Z - \mset{0}$, and let $p_1, ..., p_n$ be a complete list of prime integers dividing into $a$ and/or $b$. Furthermore let

\begin{align*}
    a & = \varepsilon_a p_{1}^{r_1} \cdot ... \cdot p_{n}^{r_n} \\
    b & = \varepsilon_b p_{1}^{s_1} \cdot ... \cdot p_{n}^{s_n}
\end{align*}

be the the respective prime factorisations of $a$ and $b$ (Note that $r_j, s_j \geq 0$ for all $j \in \mset{1, ..., n}$ but some might be 0), then

$$
    gcd(a, b) = p_{1}^{min\{r_1, s_1\}} \cdot ... \cdot p_{n}^{min\{r_n, s_n\}}
$$

\subsubsection*{2}

Given $a, b \in \Z$, we let $lcm(a, b)$ denote the smallest =positive integer divisible by both a and b, called the least common multiple of a and b.

With a and b as above, we have

$$
    lcd(a, b) = p_{1}^{max\{r_1, s_1\}} \cdot ... \cdot p_{n}^{max\{r_n, s_n\}}
$$

\pagebreak


\section{Exercises}



\end{document}
