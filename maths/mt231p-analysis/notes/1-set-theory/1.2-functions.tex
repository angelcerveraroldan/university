\subsection{Functions}

\defBox{Function}{
    Let $A$ and $B$ bet two sets, then a function $f: A \rightarrow B$ is a 
    set of ordered pairs in $A \times B$ with the property that if $(a, b)$ and $(a, b')$ 
    are elements of $f$, then $b = b'$
}

With the above definition, the domain of $f$ is the set of all possible values for $a$. 
We can also set restrictions on functions. Take $C \subset A$, then we can restrict $f$ to $C$:

$$
    f|_C: C \rightarrow B \quad f|_C(x) = f(a) \quad \forall a \in C
$$

Some functions have no formula, some do. Some functions can even have multiple branches, for example, take:

\begin{equation*}
    f =
     \begin{cases}
     x+1 & \text{for } x \leq 0 \\
     x-1 & \text{for } x > 0
     \end{cases}
\end{equation*}

Fuctions can also be composed, take $f: A \rightarrow B$, and $g: C \rightarrow D$, then the composition of $f$ and $g$ is 
written as $(g \circ f) (x) = g(f(x))$ for all x where $f(x) \in C$.

\defBox{Injections / One-to-One}{
    A function $f: A \rightarrow B$ is one to one if for any pair $a, b \in A$ 
    $f(a) = f(b)$ only if $a = b$. 

    In other words, there are no two \textbf{different} inputs that will yield the same output
}

\defBox{Surjections / Onto}{
    A function $f: A \rightarrow B$ is onto if $\forall b \in B$, $\exists a \in A$ such that $f(a) = b$
}

\defBox{Bijections}{
    A function that is both injective and surjective
}

\thmBox{}{
    Let $f: A \rightarrow B$ be a bijection. Then its inverse, $g = f^{-1}$ exists. 
}

We must define $g: B \rightarrow A$ such that $(g \circ f) (a) = a \quad \forall a \in A$, furthermore,  
$(f \circ g) (b) = b \quad \forall b \in B$

Let $b \in B$, because $f$ is onto, then $\exists a \in A$ such that $f(a) = b$. Also, $f$ is one to one, so 
$a$ is a unique elment. We define $g(b) = a$. 

Then $g(f(a)) = g(b) = a$ for all $a \in A$. And $f(g(b)) = f(a) = b$ for all $b \in B$. Therefore $g = f^{-1}$


\thmBox{}{
    Let $f: A \rightarrow B$ be a bijection, and $g: B \rightarrow C$ be a bijection. Then $g \circ f: A \rightarrow C$ is a 
    bijection. 
}

To show that $g \circ f$ is a bijection, we need to show that it is one to one, and that it is onto. 

Take $a_1, a_2 \in A$, if $g(f(a_1)) = g(f(a_1))$, then because $g$ is one to one, $f(a_1) = f(a_2)$, because $f$ is also one to one, 
then $a_1 = a_2$, therefore $g \circ f$ is one to one. 

Take $c \in C$, because $g$ is onto, then $\exists b \in B$ such that $g(b) = c$, and because $f$ is also onto, then 
$\exists a \in A$ such that $f(a) = b$, therefore, $g \circ f$ is onto. 

Because we have shown that $g \circ f$ is onto and one to one, we have proven that $g \circ f$ is a bijection. 
\pagebreak