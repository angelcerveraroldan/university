\documentclass[12pt]{article} % use the article class

\input{../../../preamble/homework/packages.tex} % imports the packages.tex file from the preamble directory
\input{../../../preamble/homework/environments.tex} % imports the environments.tex file from the preamble directory
\renewcommand{\part}[1]{\textbf{\large Part \Alph{partCounter}}\stepcounter{partCounter}\\} % part macro
\newcommand{\solution}{\textbf{\large Solution}} % solution macro

\newmdenv[
    backgroundcolor=gray!20,
    skipabove=\topsep,
    skipbelow=\topsep,
]{grayBoxed}

\newmdenv[
    backgroundcolor=blue!20,
    skipabove=\topsep,
    skipbelow=\topsep,
]{blueBoxed}

% General writing
\newcommand{\bracket}[1]{\left(#1\right)} % for automatic resizing of brackets
\newcommand{\sbracket}[1]{\left[#1\right]} % for automatic resizing of square brackets
\newcommand{\mset}[1]{\left\{#1\right\}} % for automatic resizing of curly brackets
\newcommand{\defeq}{\coloneqq} % the "defined as" command
\newcommand{\RNum}[1]{\uppercase\expandafter{\romannumeral #1\relax}} % uppercase roman numerals

\newcommand{\rednote}[1]{{\color{red} #1}} % for red text
\newcommand{\bluenote}[1]{{\color{blue} #1}} % for blue text
\newcommand{\greennote}[1]{{\color{green} #1}} % for green text

% Blackboard Maths Symbols
\newcommand{\N}{\mathbb{N}} % natural numbers
\newcommand{\Q}{\mathbb{Q}} % rational numbers
\newcommand{\Z}{\mathbb{Z}} % integers
\newcommand{\R}{\mathbb{R}} % real numbers
\newcommand{\C}{\mathbb{C}} % complex numbers
\newcommand{\E}{\mathbb{E}} % expectation operators

% Aesthetic
\newcommand{\thmBox}[2]{
    \begin{tcolorbox}[colback=blue!5!white,colframe=blue!50!black,
            colbacktitle=blue!90!black,title=Theorem #1]
        #2
    \end{tcolorbox}
}

\newcommand{\lmBox}[2]{
    \begin{tcolorbox}[colback=blue!5!white,colframe=blue!50!black,
            colbacktitle=blue!50!black,title=Lemma #1]
        #2
    \end{tcolorbox}
}

\newcommand{\defBox}[2]{
    \begin{tcolorbox}[colback=red!5!white,colframe=red!50!black,
            colbacktitle=red!60!black,title=Definition #1]
        #2
    \end{tcolorbox}
} % imports the macros.tex file from the preamble directory
\input{../../../preamble/homework/settings.tex} % imports the settings.tex file from the preamble directory

\title{
    \vspace{2in}
        \textmd{\textbf{MT231 - Analysis 1}}\\
    \vspace{1in}
    \textmd{\textbf{Homework \#1}}\\
    \vspace{1in}
}

\author{
    \hmwkAuthorName\\
    \hmwkStudentnum
}

\date{}

\begin{document}

\maketitle

\pagebreak

\begin{homeworkProblem}

    \begin{grayBoxed}
        Prove that for any sets $A$ and $B$, the following holds

        $$
            A = (A \cap B) \cup (A \setminus B)
        $$
    \end{grayBoxed}

    To prove this, we will show that any element $x \in A$ will also be a part of $(A \cap B) \cup (A \setminus B)$

    For any element, it can either be in $B$ or not be in $B$, therefore:

    \begin{align*}
        x  & \in A                                                                       \\
        x  & \in A \text{ and } (x \in B \text{ or } x \not \in B)                       \\
        (x & \in A \text{ and } x \in B) \text{ or } (x \in A \text{ and } x \not \in B) \\
        (x & \in A \cap B) \text{ or } (x \in A \setminus B)                             \\
        x  & \in (A \cap B)\cup (A \setminus B)
    \end{align*}

    Therefore, if $x \in A$, then $x \in (A \cap B)\cup (A \setminus B)$.
    This shows that $A \subseteq (A \cap B)\cup (A \setminus B)$ as every element in $A$ has to be in $(A \cap B)\cup (A \setminus B)$.

    Now we need to show that every element $x \in (A \cap B)\cup (A \setminus B)$ also has to be in $A$.

    \begin{align*}
        x  & \in (A \cap B)\cup (A \setminus B)                                          \\
        x  & \in (A \cap B) \text{ or } x   \in (A \setminus B)                          \\
        (x & \in A \text{ and } x \in B) \text{ or } (x \in A \text{ and } x \not \in B) \\
        x  & \in A \text{ and } (x \in B \text{ or } x \not \in B)                       \\
    \end{align*}

    Therefore, if $x \in (A \cap B)\cup (A \setminus B)$, then $x \in A \text{ and } (x \in B \text{ or } x \not \in B)$,
    which shows that regardless of wether x is or isn't in $B$, x will be in $A$. This means that every element in
    $(A \cap B)\cup (A \setminus B)$ is in $A$. Therefore, $(A \cap B)\cup (A \setminus B) \subseteq A$.

    Because $(A \cap B)\cup (A \setminus B)$ is a subset of $A$, and $A$ is a subset of $(A \cap B)\cup (A \setminus B)$, that means that
    every element in one set must be in the other, they are therefore the same.

\end{homeworkProblem}

\pagebreak

\begin{homeworkProblem}
    \begin{grayBoxed}
        Let $a, b \in \R$, where $a < b$, find a bijection from $(a, b)$ to $(0, 1)$.
    \end{grayBoxed}

    Fist, find a bijection from $(a, b)$ to $(0, n)$ for some $n \in \R$.

    \begin{align*}
        g & : (a, b) \rightarrow (0, n)         \\
        g & : (a, b) \rightarrow (a - a, b - a) \\
        g & (x)      = x - a                    \\
    \end{align*}

    $g$ is clearly a bijection as it is a linear function. Now, find a bijection from $(0, n)$ to $(0, 1)$.

    \begin{align*}
        p: (0, b - a) & \rightarrow (0, 1) \\
        p (0)         & = 0                \\
        p (b - a)     & = 1                \\
        p (x)         & = \frac{x}{b - a}  \\
    \end{align*}

    If we combine the two functions above, we get:

    \begin{align*}
        f(x) & = p \circ g                           \\
             & = p(g(x))                             \\
             & = \frac{x - a}{b - a}                 \\
             & = \frac{1}{b - a} x - \frac{a}{b - a} \\
    \end{align*}

    Because $\frac{1}{b - a}$ is a constant, the function $f(x)$ is linear, and therefore a bijection.

\end{homeworkProblem}

\pagebreak

\begin{homeworkProblem}
    \begin{grayBoxed}
        Prove that a function $f: A \rightarrow B$ which possesses an inverse must be a bijection.
    \end{grayBoxed}
\end{homeworkProblem}

\pagebreak

\begin{homeworkProblem}
    \subsection*{Part a}

    \begin{grayBoxed}
        Consider the function $f: A \rightarrow B$. Show that setting $a_1 \sim a_2$ if $f(a_1) = f(a_2)$ defines an equivalence relation on A.
    \end{grayBoxed}


    \subsection*{Part b}

    \begin{grayBoxed}
        Identify the equivalence classes under this equivalence relation if $f: \R \rightarrow \R$ is given by $f(x) = x^2$
    \end{grayBoxed}

    \begin{align*}
        f(a)  & = f(b) \\
        a^2   & = b^2  \\
        \pm a & = b    \\
    \end{align*}

    Therefore the equivalence class of any x is:

    $$[x] = \mset{x, -x}$$

    \subsection*{Part c}

    \begin{grayBoxed}
        Identify the equivalence classes under this equivalence relation if $f: \R \rightarrow \R$ is given by $f(x) = \lfloor x \rfloor$
    \end{grayBoxed}

    There are an infinite amount of equivalence classes, all in the form $[n, n + 1)$ for some integer $n$

    $$[x] = [\lfloor x\rfloor, \lfloor x\rfloor + 1)$$

\end{homeworkProblem}

\pagebreak

\begin{homeworkProblem}
    Let $C$ be the set of counties in Ireland.

    \subsection*{Part a}

    \begin{grayBoxed}
        Give an example of an equivalence relation on $C$. What are the equivalence classes of this relation?
    \end{grayBoxed}

    An example of an equivalence relation on $C$ would be $a \sim b$ if the first letter of $a$ is the same as the first letter of $b$.

    This is an equivalence relation as it fulfils all 3 rules:
    \begin{enumerate}
        \item $a \sim a$ since a will always have the same name as a, a will always have the same first letter as a.
        \item if $a \sim b$, then $b \sim a$ since if a and b have the same first letter, then b and a will also have the same  first letter.
        \item if $a \sim b$, and $b \sim c$, then $a \sim c$. If a and b have the same first letter, and b and c have the same first letter, then c and a must have the same first letter, and therefore, $a \sim c$.
    \end{enumerate}


    \subsection*{Part b}

    \begin{grayBoxed}
        Give another example of an equivalence relation on $C$.
    \end{grayBoxed}

    Another example of an equivalence relation on $C$ would be $a \sim b$ if a and b have the same number of houses\dots

    This is an equivalence relation as it fulfils all 3 rules:
    \begin{enumerate}
        \item $a \sim a$ since a will always have the same houses as a.
        \item if $a \sim b$, then $b \sim a$ since if a and b have the same number of houses, then b and a will also have the same number of houses.
        \item if $a \sim b$, and $b \sim c$, then $a \sim c$. If a and b have the same number of houses, and b and c have the same number of houses, then c and a must have the same number of houses, and therefore, $a \sim c$.
    \end{enumerate}

    \subsection*{Part c}

    \begin{grayBoxed}
        Give an example of a relation on $C$ which is not an equivalence relation.
    \end{grayBoxed}

    An example of a relation on $C$ which it not an equivalence relation would be $a \sim b$ if $a$ and $b$ border each other.

    This is not an equivalence relation as Waterford and Cork border each other, therefore Waterford $\sim$ Cork.
    Cork also borders Kerry therefore Cork $\sim$ Kerry. However Waterford $\not \sim$ Kerry.

    This breaks the transitive property of equivalence relations.

\end{homeworkProblem}

\end{document}
