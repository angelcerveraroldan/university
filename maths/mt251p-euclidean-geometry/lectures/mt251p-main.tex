\documentclass[12pt]{article} % use the article class

\input{../../../preamble/packages.tex} % imports the packages.tex file from the preamble directory
\input{../../../preamble/homework/environments.tex} % imports the environments.tex file from the preamble directory
\renewcommand{\part}[1]{\textbf{\large Part \Alph{partCounter}}\stepcounter{partCounter}\\} % part macro
\newcommand{\solution}{\textbf{\large Solution}} % solution macro

\newmdenv[
    backgroundcolor=gray!20,
    skipabove=\topsep,
    skipbelow=\topsep,
]{grayBoxed}

\newmdenv[
    backgroundcolor=blue!20,
    skipabove=\topsep,
    skipbelow=\topsep,
]{blueBoxed}

% General writing
\newcommand{\bracket}[1]{\left(#1\right)} % for automatic resizing of brackets
\newcommand{\sbracket}[1]{\left[#1\right]} % for automatic resizing of square brackets
\newcommand{\mset}[1]{\left\{#1\right\}} % for automatic resizing of curly brackets
\newcommand{\defeq}{\coloneqq} % the "defined as" command
\newcommand{\RNum}[1]{\uppercase\expandafter{\romannumeral #1\relax}} % uppercase roman numerals

\newcommand{\rednote}[1]{{\color{red} #1}} % for red text
\newcommand{\bluenote}[1]{{\color{blue} #1}} % for blue text
\newcommand{\greennote}[1]{{\color{green} #1}} % for green text

% Blackboard Maths Symbols
\newcommand{\N}{\mathbb{N}} % natural numbers
\newcommand{\Q}{\mathbb{Q}} % rational numbers
\newcommand{\Z}{\mathbb{Z}} % integers
\newcommand{\R}{\mathbb{R}} % real numbers
\newcommand{\C}{\mathbb{C}} % complex numbers
\newcommand{\E}{\mathbb{E}} % expectation operators

% Aesthetic
\newcommand{\thmBox}[2]{
    \begin{tcolorbox}[colback=blue!5!white,colframe=blue!50!black,
            colbacktitle=blue!90!black,title=Theorem #1]
        #2
    \end{tcolorbox}
}

\newcommand{\lmBox}[2]{
    \begin{tcolorbox}[colback=blue!5!white,colframe=blue!50!black,
            colbacktitle=blue!50!black,title=Lemma #1]
        #2
    \end{tcolorbox}
}

\newcommand{\defBox}[2]{
    \begin{tcolorbox}[colback=red!5!white,colframe=red!50!black,
            colbacktitle=red!60!black,title=Definition #1]
        #2
    \end{tcolorbox}
} % imports the macros.tex file from the preamble directory
\input{../../../preamble/homework/settings.tex} % imports the settings.tex file from the preamble directory
 

\title{
    \vspace{2in}
        \textmd{\textbf{MT251 - Euclidean Geometry}}\\
    \vspace{1in}
    \textmd{\textbf{Lecture Notes}}\\
    \vspace{1in}
}

\author{
    \hmwkAuthorName\\
    \hmwkStudentnum
}

\date{24th October 2022}

\begin{document}

\maketitle

\pagebreak

Remakrk 6. Suppose we are in three dimensional space given by $\R^3 = \mset{(x, y, x) \ | \ x, y, z \in \R}$

and suppose that $A, B \in \R^3$

The vector $\vec{u}$ = $\vec{AB}$ can be written as: GIVE A AND B (X, Y, Z) VALUES AND ADD ALGEBRAICALLY

\subsection*{Theorem 2}
\begin{grayBoxed}
    Suppose $\vec{v} = v_1 i + v_2 j + v_3 k$ and $\vec{w} = w_1 i + w_2 j + w_3 k$ and $t \in \R$.
    \begin{enumerate}
        \item $\vec{v} + \vec{w} = (v_1 + w_1) i + (v_2 + w_2) j + (v_3 + w_3) k$
        \item $t \vec{v} = (t v_1 i) +(t v_2 j) + (t v_3) k$
        \item $||\vec{v}|| = \sqrt{v_1^2 + v_2^2 + v_3^2}$
        \item $\vec{v} \cdot \vec{w} = v_1w_1 + v_2 w_2 + v_3 w_3$
        \item $\vec{v} \cdot \vec{w} = ||\vec{v}|| ||\vec{w}|| \cos{\theta}$, $v, w \not = 0$, where $0 \leq \theta \leq \pi$
    \end{enumerate}
\end{grayBoxed}

\subsection{Definition 2}
Two non 0 vector are said to be perpendicular if the angle between them is $\frac{\pi}{2}$.

The 0 vector is perpendicular to every vector.


\subsection{Definition 3}
Suppose $\vec{u} = u_1i + u_2j + u_3k$ and $\vec{w} = w_1i + w_2j + w_3k$, the cross product of $\vec{u}$ and $\vec{w}$ is denoted by
$\vec{u} \cross \vec{w}$ and is defined by:

$$
    \vec{u} \cross \vec{w} = (u_2 w_3 - u_3 w_2)i + (u_3 w_1 - u_1 w_3)j + (u_1 w_2 - u_2 w_1)k
$$

\subsection{Remark 8}
If $\vec{u}$ and $\vec{w}$ are vectors in $\R^3$, then $u \cross w$ gives a vector which is perpendicular to both $\vec{u}$ and $\vec{w}$.

\pagebreak

\subsection{Example 6}

\begin{grayBoxed}
    Consider the vectors $\vec{u} = i + 2j + k$ and $\vec{w} = 3i + j + 2k$. Find the angle $\theta$ between $\vec{u}$ and $\vec{w}$.
\end{grayBoxed}

\begin{align*}
    \vec{u} \cdot \vec{w} & = 7                               \\
    ||\vec{u}||           & = \sqrt{6}                        \\
    ||\vec{w}||           & = \sqrt{14}                       \\
    7                     & = \sqrt{6} \sqrt{14} \cos{\theta} \\
    \cos{\theta}          & = \frac{7}{\sqrt{84}}             \\
    \theta                & = \arccos{\frac{7}{\sqrt{84}}}
\end{align*}

\section{Vectors in $\R^n$}

\subsection{Definition 4}
For $n \geq 1$ we define

$$
    \R^n = \mset{(x_1, x_2, ...) \ | \ x_i \in \R \text{ for } 1 \leq i \leq n}
$$


\subsection*{Theorem 3}
\begin{grayBoxed}
    Suppose $(x_1, x_2, ..., x_n)$ and $(y_1, y_2, ..., y_n) \in \R^n$.
    \begin{enumerate}
        \item $(x_1, x_2, ..., x_n) + (y_1, y_2, ..., y_n) = (x_1 + y_1, x_2 + y_2, ..., x_n + y_n)$
        \item $t (x_1, x_2, ..., x_n) = (tx_1, tx_2, ..., tx_n)$
        \item $||(x_1, x_2, ..., x_n)|| = \sqrt{x_{1}^2 + x_{2}^2 + ... + x_{n}^2}$
        \item $\vec{x} \cdot \vec{y} = (x_1  y_1, x_2  y_2, ..., x_n y_n)$
        \item $\vec{v} \cdot \vec{w} = ||\vec{v}|| ||\vec{w}|| \cos{\theta}$, $v, w \not = (0, 0, ..., 0)$, where $0 \leq \theta \leq \pi$
    \end{enumerate}
\end{grayBoxed}

\end{document}
